\documentclass[11pt]{article}
\usepackage{amsfonts,amssymb,amsthm,eucal,amsmath}
\usepackage{graphicx}
\usepackage[T1]{fontenc}
\usepackage{latexsym,url}
\usepackage{array}
\usepackage{subfig}
\usepackage{comment}
\usepackage{color}
\usepackage{tikz}
\usepackage{cprotect}
\usepackage{hyperref}
\usepackage[nameinlink,noabbrev]{cleveref}

\hypersetup{colorlinks=true,linkcolor=blue}
\newcommand{\myspace}{\vspace{.1in}\noindent}
\newcommand{\mymyspace}{\vspace{.1in}}
\usepackage[inner=30mm, outer=30mm, textheight=225mm]{geometry}
\creflabelformat{equation}{#2(#1)#3}
\crefname{equation}{}{}
\Crefname{equation}{}{}

\newtheorem{theorem}{Theorem}[section]
\newtheorem{prop}[theorem]{Proposition}
\newtheorem{corollary}[theorem]{Corollary}
\newtheorem{defn}[theorem]{Definition}
\newtheorem{notn}[theorem]{Notation}
\newtheorem{cond}[theorem]{Condition}
\newtheorem{lemma}[theorem]{Lemma}
\newtheorem{ex}[theorem]{Example}
\newtheorem{rmk}[theorem]{Remark}
\newcommand{\co}{\negthinspace :}
\newcommand{\N}{\mathbb{N}}
\newcommand{\Z}{\mathbb{Z}}
\newcommand{\R}{\mathbb{R}}
\newcommand{\E}{\mathbb{E}}
\newcommand{\C}{\mathbb{C}}
\newcommand{\CP}{\mathbb{CP}}
\newcommand{\PSL}{\mathrm{PSL}_2(\mathbb{C})}
\newcommand{\area}{\operatorname{area}}
\newcommand{\diag}{\operatorname{diag}}
\newcommand{\nt}{\negthinspace}
\newcommand{\TODO}{{\color{red} TODO}}

\title{Do hyperbolic tilings isometrically embed in $\R^3$?}
\author{Geoffrey Irving\thanks{Email: \{irving\}@naml.us, Otherlab, San Francisco, CA, United States}}
\date{Version 1, \today}

\begin{document}
\maketitle

\begin{abstract}
Do hyperbolic triangulations and other tilings isometrically embed in 3-space?
Henry says this is unresolved, and may have been considered by Thurston.
However, it seems like there is an amazing amount of slack in the result, so
here are some thoughts.
\end{abstract}

\section{Hyperbolic triangulations}

Let $T$ be the infinite valence 7 hyperbolic triangulation where each edge has length 1.
Let $T_n$ be the radius $n$ portion of $T$ centered at a fixed origin vertex $0$.  Due to negative
curvature, the number of triangles in $T_n$ is at least $\alpha^n$ for some $\alpha > 1$.

\section{Isometric embeddings}

Assume $T$ has an isometric embedding into $\R^3$, where each triangle is mapped to a unit length flat
equilateral triangle.

Given a vertex-triangle pair $(v,t)$ with $v \in t \in T$, let $f(v,t) \in \R^9$ be the three vertices
of triangle $t$ in order starting at $v$ (all triangles are oriented since the hyperbolic plane is
orientable).  Consider the image $f(T_n) \subset \R^9$: since $T_n$ has radius $\le n$, $f(T_n)$ must
lie in ball of radius $3n+O(1)$, which therefore has volume at most $\beta (3n+O(1))^9$ where $\beta$
is the volume of the unit 9-sphere.

Now define
$$d(n,k) = \sup \left\{ d | \exists q \in \R^9, vt_1 \ne \ldots \ne vt_k \in T_n . \left|f(vt_i) - q\right| < d \right\}$$
Placing a radius $d(n,k)/2$ 9-sphere around each $f(vt)$ gives the volume inequality
$$\beta d(n,k)^9 \alpha^n / k < \tau \beta (3n+O(1))^9$$
where the extra factor of $\tau$ is needed to weed out parts of the little balls that go outside
the large ball and the fact of two in $d(n,k)/2$.  Solving (and changing $\tau$), we have
$$d(n,k) < \tau k^{1/9} n \alpha^{-n}.$$

If we instead fix a separation distance $d(n,k) = \epsilon$, solving for $k$ gives
$$k > \tau_\epsilon \frac{\alpha^n}{n^9}$$
Thus, for an arbitrarily small notion of closeness, there is an exponentially large set of close triangles
at most a linear geodesic distance apart.  Call the largest such set $P = P(\epsilon,n)$.
Consider minimal geodesic paths $\gamma_{ij}$ between all pairs of centers of triangles in $P$.

The next thing to do
is to analyze the space swept out by the triangles along these paths, in an attempt to prove that the normals
along at least most of the paths stay close to the source and destination normals (which are all close).
And the next thing after that is to apply some sort of intermediate value theorem to show that this is impossible:
either there must be intersections between the triangles as they change height, or we've almost embedded a chunk of
hyperbolic space into $\R^2$.

I think the second of the above conjectures is easier than the first.  Our plan of attack on the first is to
define some notion of nonintersecting volume which becomes significantly nonzero if a geodesic path rotates
out of the plane, and then prove that this volume blows up impossibly fast.  If we succeed, it will likely
render the above weak calculation superfluous.

The following is rough: Consider what happens when a dihedral angle between two adjacent triangles is bounded
away from both 0 and $pi$ (neither flat not completely folded).  If we restrict to only considering similarly
shaped hinges (which we can build into our volume measure), there can only be infinitely many stacked copies
if they are nested.  Thus, our path space is either locally sparse or locally very nested with a local total order.
The problem is that this total order is unlikely to conveniently propagate around since adjacent hinges necessarily
keep flipping.  So this direction may go nowhere.

More roughness: consider four close triangles $a,b,c,d$ in order from lowest to highest along their roughly
shared normal $n$, and the geodesic paths $\gamma_{ac}$ and $\gamma_{db}$ both parameterized on $s \in [0,1]$.
Consider the function
$$h(s) = (n_{ac}(s) + n_{bd}(s)) \cdot \left(\gamma_{ac}(s) - \gamma_{bd}(s)\right)$$
where $n_{ac}(s)$ and $n_{bd}(s)$ are (roughly) the normals to $\gamma_{ac}$ and $\gamma_{bd}$.  $h(s)$ can
be made continuous if we add a delay at each edge crossing to slowly change normal.
We have $h(0) < 0$, $h(1) > 0$, so there must be a zero crossing $h(s^*) = 0$.  What is happening at this
crossing?  There are several (rough) options:
\begin{enumerate}
\item The two points lie in the same triangle of the same surface (or say within geodesic distance 2 on the surface).
\item The two points are in different ``simplex classes'': triangle interior, edge interior, vertex.
\item The two points are both in the interior of their triangles, and are far apart (necessary to cross).
      Note that they may have opposite normals (is this a problem?).
\item Both points are on the interior of their edges, but are on different looking hinges.  If the hinges
      are sufficiently similar, $h(s)$ can't change sign.
\item Both points are very close to a vertex.  I believe this still rules out a crossing.
\end{enumerate}
It seems like all but the first case should produce a nontrivial nonintersecting volume measure assigned to
this crossing event.  Actually I don't know whether these events can be arranged to be nonintersecting so
as to produce a contradiction.  In any case, we have another problem: if we have an exponential number of
close triangles in $T_n$, is there at least an exponential number of pairs of pairs whose geodesics don't
cross, and whose heights have the right ordering?

\begin{comment}
\begin{lemma}
Let $P \subset B_r$ be a set of points separated by distance at least 1 in a hyperbolic disk of radius $r$
with $|P| = \Theta(\beta^r)$ for some $\beta > 1$.  Totally order $P$.  Then there exists $\kappa > 1$
and an ordered subset $p_1, \ldots, p_{2n} \in P$ with $n = \Theta(\kappa^r)$ s.t.\ the geodesics

Let B(r) be a ball of radius r in the hyperbolic plane, which has
volume Theta(A^r).  Choose Theta(A^r/r^9) points x_i pairwise
separated by at least distance 1, and total order them somehow: x_1 <
x_2 < ...  Does there exist an absolute constant B > 1 s.t. there
always exists a totally ordered subsequence y_1 < ... < y_(2n) of size
Theta(B^r) s.t. the n geodesics from y_i to y_(i+n) are pairwise
separated by at least 1?

Intuition: x_i are the centers of triangles that are very close
together, and we want to arrive at a contradiction by analyzing how
the various geodesics between x_i to x_j "change height order".  If
the above was true, we wouldn't have to worry about the geodesics
dodging contradictions by simply sharing triangles.
\end{comment}

\end{document}
