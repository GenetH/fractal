\documentclass[11pt]{article}
\usepackage{amsfonts,amssymb,amsthm,eucal,amsmath}
\usepackage{graphicx}
\usepackage[T1]{fontenc}
\usepackage{latexsym,url}
\usepackage{array}
\usepackage{subfig}
\usepackage{comment}
\usepackage{color}
\usepackage{tikz}
\usepackage{cprotect}
\usepackage{hyperref}
\usepackage[nameinlink,noabbrev]{cleveref}

\hypersetup{colorlinks=true,linkcolor=blue}
\newcommand{\myspace}{\vspace{.1in}\noindent}
\newcommand{\mymyspace}{\vspace{.1in}}
\usepackage[inner=30mm, outer=30mm, textheight=225mm]{geometry}
\creflabelformat{equation}{#2(#1)#3}
\crefname{equation}{}{}
\Crefname{equation}{}{}

\newtheorem{theorem}{Theorem}[section]
\newtheorem{prop}[theorem]{Proposition}
\newtheorem{corollary}[theorem]{Corollary}
\newtheorem{defn}[theorem]{Definition}
\newtheorem{notn}[theorem]{Notation}
\newtheorem{cond}[theorem]{Condition}
\newtheorem{ex}[theorem]{Example}
\newtheorem{rmk}[theorem]{Remark}
\newcommand{\co}{\negthinspace :}
\newcommand{\N}{\mathbb{N}}
\newcommand{\Z}{\mathbb{Z}}
\newcommand{\R}{\mathbb{R}}
\newcommand{\E}{\mathbb{E}}
\newcommand{\C}{\mathbb{C}}
\newcommand{\CP}{\mathbb{CP}}
\newcommand{\PSL}{\mathrm{PSL}_2(\mathbb{C})}
\newcommand{\area}{\operatorname{area}}
\newcommand{\diag}{\operatorname{diag}}
\newcommand{\nt}{\negthinspace}
\newcommand{\TODO}{{\color{red} TODO}}

\title{Do hyperbolic tilings isometrically embed in $\R^3$?}
\author{Geoffrey Irving\thanks{Email: \{irving\}@naml.us, Otherlab, San Francisco, CA, United States}}
\date{Version 1, \today}

\begin{document}
\maketitle

\begin{abstract}
Do hyperbolic triangulations and other tilings isometrically embed in 3-space?
Henry says this is unresolved, and may have been considered by Thurston.
However, it seems like there is an amazing amount of slack in the result, so
here are some thoughts.
\end{abstract}

Let $T$ be the infinite valence 7 hyperbolic triangulation where each edge has length 1.
Let $T_n$ be the radius $n$ portion of $T$ centered at a fixed origin vertex $0$.  Due to negative
curvature, the number of triangles in $T_n$ is at least $\alpha^n$ for some $\alpha > 1$.

Assume $T$ has an isometric embedding into $\R^3$, where each triangle is mapped to a unit length flat
equilateral triangle.

Given a vertex-triangle pair $(v,t)$ with $v \in t \in T$, let $f(v,t) \in \R^9$ be the three vertices
of triangle $t$ in order starting at $v$ (all triangles are oriented since the hyperbolic plane is
orientable).  Consider the image $f(T_n) \subset \R^9$: since $T_n$ has radius $\le n$, $f(T_n)$ must
lie in ball of radius $3n+O(1)$, which therefore has volume at most $\beta (3n+O(1))^9$ where $\beta$
is the volume of the unit 9-sphere.

Now define
$$d(n,k) = \sup \left\{ d | \exists q \in \R^9, vt_1 \ne \ldots \ne vt_k \in T_n . \left|f(vt_i) - q\right| < d \right\}$$
Placing a radius $d(n,k)/2$ 9-sphere around each $f(vt)$ gives the volume inequality
$$\beta d(n,k)^9 \alpha^n / k < \tau \beta (3n+O(1))^9$$
where the extra factor of $\tau$ is needed to weed out parts of the little balls that go outside
the large ball and the fact of two in $d(n,k)/2$.  Solving (and changing $\tau$), we have
$$d(n,k) < \tau k^{1/9} n \alpha^{-n}.$$

If we instead fix a separation distance $d(n,k) = \epsilon$, solving for $k$ gives
$$k > \tau_\epsilon \frac{\alpha^n}{n^9}$$
Thus, for an arbitrarily small notion of closeness, there is an exponentially large set of close triangles
at most a linear geodesic distance apart.  Call the largest such set $P = P(\epsilon,n)$.
Consider minimal geodesic paths $\gamma_{ij}$ between all pairs of centers of triangles in $P$.

The next thing to do
is to analyze the space swept out by the triangles along these paths, in an attempt to prove that the normals
along at least most of the paths stay close to the source and destination normals (which are all close).
And the next thing after that is to apply some sort of intermediate value theorem to show that this is impossible:
either there must be intersections between the triangles as they change height, or we've almost embedded a chunk of
hyperbolic space into $\R^2$.

I think the second of the above conjectures is easier than the first.  Our plan of attack on the first is to
define some notion of nonintersecting volume which becomes significantly nonzero if a geodesic path rotates
out of the plane, and then prove that this volume blows up impossibly fast.  If we succeed, it will likely
render the above weak calculation superfluous.

The following is rough: Consider what happens when a dihedral angle between two adjacent triangles is bounded
away from both 0 and $pi$ (neither flat not completely folded).  If we restrict to only considering similarly
shaped hinges (which we can build into our volume measure), there can only be infinitely many stacked copies
if they are nested.  Thus, our path space is either locally sparse or locally very nested with a local total order.
The problem is that this total order is unlikely to conveniently propagate around since adjacent hinges necessarily
keep flipping.  So this direction may go nowhere.

\end{document}
